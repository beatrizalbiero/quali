\chapter{Histórico Escolar}
\label{ch:01-historicoescolar}

\section{Disciplinas Cursadas}

\textbf{\Large Disciplinas com Direito a Crédito}\\

\begin{table}[ht!]
\center
\begin{tabular}{llll}
\textbf{Sigla}       & \textbf{Nome da Disciplina}                                         & \textbf{Créditos} & \textbf{Conceito Final} \\
\midrule
FLL5121 & Tópicos em Processamento de Linguagem e Funções Cognitivas & 8        & A              \\
FLF5221 & Filosofia da Ciência (A Dimensão Social da Racionalidade)   & 8        & A              \\
FLL5133 & Linguística Computacional                                  & 8        & A              \\
\midrule
            & \textbf{Total de Créditos}                                          & \textbf{24}       &               
\end{tabular}
\caption{Histórico Escolar}
\label{tab:historico}
\end{table}

\textbf{\Large Cursos de Extensão Universitária}\\

\begin{table}[h]
\begin{tabular}{ll}
\textbf{Nome da Disciplina}                 & \textbf{Conceito Final} \\
\midrule
Neurociência Cognitiva e Filosofia da Mente & A                      
\end{tabular}
\caption{Histórico em Cursos de Extensão}
\label{tab:cursosextra}
\end{table}

\section{Tópicos em Processamento de Linguagem e Funções Cognitivas}

\textbf{\Large Descrição da Disciplina}\\

\textbf{Docente Responsável}\\
\hspace*{1.6em}Felipe Venâncio Barbosa\\

\textbf{Objetivos}\\
\hspace*{1.6em}O objetivo geral desta disciplina é discutir aspectos do processamento da linguagem e as suas relações com as funções cognitivas em línguas orais e sinalizadas com e sem a presença de atipias. Fornecer ao aluno subsídios para a análise das interferências das habilidades cognitivas nas línguas orais e de sinais e do impacto da aquisição adequada de língua no desenvolvimento das habilidades cognitivas.\\

\textbf{Justificativa}\\
\hspace*{1.6em}Os estudos das relações entre cognição e língua são de extrema importância para a compreensão do processamento da linguagem em si. O impacto da aquisição adequada de língua e desenvolvimento de linguagem no desenvolvimento cognitivo tem sido foco de pesquisadores de diversas áreas no mundo. Os resultados das pesquisas publicadas a partir de análises deste tipo promovem benéficas contribuições práticas para pessoas acometidas por distúrbios que causam alterações no processamento da linguagem, pois instrumentalizam a prática profissional de diversas áreas de interface com os estudos da linguagem além de promover discussões e esclarecimentos a respeito do próprio funcionamento da língua.\\

\textbf{Forma de Avaliação}\\
\hspace*{1.6em}Dissertação.\\

\textbf{Relevância para a Pesquisa}

O tema de processamento de linguagem está intrinsecamente relacionado ao tema da minha pesquisa. Compreender o processo de aquisição de linguagem, bem como a sua relação com outros processos cognitivos está alinhado com o meu objetivo de obter um melhor entendimento a respeito do aprendizado linguístico humano.\\

\textbf{Trabalho Final}

Como trabalho final, entreguei uma dissertação com o título de \textit{"Processamento de Linguagem Natural: Desafios"}. Nessa dissertação exploro dificuldades computacionais em tarefas linguísticas consideradas simples para qualquer falante. Aponto questões como Inferência em Linguagem Natural (Natural Language Inference) e modelos de linguagem com memórias longas. Concluo com um resumo de tarefas que já são consideradas bem sucedidas no campo, porém atento ao fato de ainda haver uma vasta área a ser estudada e explorada antes dos projetos linguístico-computacionais serem considerados realmente eficientes.


%faltou a disciplina extra que eu cursei com a fabiana

\section{Filosofia da Ciência (A Dimensão Social da Racionalidade)}

\textbf{\Large Descrição da Disciplina}\\

\textbf{Docente Responsável}\\
\hspace*{1.6em}Caetano Ernesto Plastino\\

\textbf{Objetivos}\\
\hspace*{1.6em}Examinar a dimensão social da racionalidade, tanto teórica como prática, fazendo uso de exemplos paradigmáticos encontrados na vida diária e na ciência. Mostrar as limitações do modelo clássico de crença ou decisão racional centrado no indivíduo. Compreender a natureza social do conhecimento humano e do processo de escolha, a partir de recentes estudos de epistemologia coletiva e de recursos da teoria dos jogos.\\

\textbf{Forma de Avaliação}\\
\hspace*{1.6em}Apresentação de seminário e dissertação.\\

\textbf{Relevância para a Pesquisa}

O impacto social gerado pelo desenvolvimento de pesquisas em inteligência artificial é uma discussão contemporânea e urgente à medida que a tecnologia disponível já alcançou patamares em que se faz necessário entender os problemas éticos e morais que a presença destes novos itens tecnológicos podem trazer para a vivência humana.\\

\textbf{Trabalho Final}\\
Como trabalho final, entreguei uma dissertação intitulada de \textit{"Esboços de Diretrizes da Práxis Epistemológica"} onde problematizo a experiência do sujeito diante das novas revoluções tecnológicas e aponto os problemas de percepção, as consequências desses problemas perante a rápida proliferação de estímulos supernormais proveniente dessas novas tecnologias e da importância do desenvolvimento de uma postura atualizada diantes destes novos paradigmas.

\section{Linguística Computacional}
\textbf{\Large Descrição da Disciplina}\\

\textbf{Docentes Responsáveis}\\
\hspace*{1.6em}Marcos Fernando Lopes
\hspace*{1.6em}Marcelo Barra Ferreira\\

\textbf{Objetivos}\\
\hspace*{1.6em}Introduzir conceitos, métodos e técnicas utilizados no processamento de linguagem natural (PLN) por computadores em diversos níveis de análise: fonológico, morfológico, sintático, semântico, pragmático e discursivo. Apresentar modelos e algoritmos usados na linguística computacional, enfatizando suas aplicações científicas e tecnológicas. Capacitar o aluno a escrever e avaliar programas de computador que desempenhem tarefas ligadas ao PLN.\\

\textbf{Justificativa}\\
\hspace*{1.6em}A existência atual de computadores capazes de armazenar e processar de maneira rápida e eficiente quantidades cada vez maiores de dados tem conferido à linguística computacional um papel científico e tecnológico inegável. Por um lado, a possibilidade de se construir e analisar automaticamente vastos corpora linguísticos tem proporcionado uma base de dados robusta valiosíssima, contra a qual se pode avaliar modelos teóricos linguísticos de diversas estirpes e em diversos níveis. Por outro lado, aplicações práticas de grande utilidade como corretores ortográficos, classificadores de documentos, analisadores de sentimentos, tradutores automáticos, para citar apenas alguns exemplos, são resultados diretos dos esforços interdisciplinares que caracterizam o campo. Tudo isso, sem dúvida alguma, abre ao linguista que domine os conceitos e técnicas que o curso abordará uma gama de possibilidades de inserção no mercado de trabalho tanto dentro quanto fora do mundo acadêmico.\\

\textbf{Forma de Avaliação}\\
\hspace*{1.6em}Provas + Trabalho Final\\

\textbf{Relevância para a Pesquisa}

Essa disciplina foi fundamental para o meu desenvolvimento como pesquisadora e desenvolvedora na área de Linguística Computacional. Neste curso pude, além de reforçar os meus conhecimentos em lógica de programação, compreender boas práticas em PLN, em especial, técnicas para análise, limpeza de corpus e classificação de texto. 

Além disso, o curso abordou o tema de Modelos de Linguagem (\textit{n-gramas}), conceito que foi de suma importância para o desenvolvimento deste trabalho.\\

\textbf{Trabalho Final}

Como trabalho final, desenvolvi um sistema de diálogo automático (\textit{chatbot}) simples, especializado em atender dúvidas comuns de usuários a respeito da secretaria da pós-graduação (horário de funcionamento, telefone de contato e e-mails, endereço, entre outros). 

Foi entregue também um relatório em formato de artigo intitulado de "\textit{Modelos de Classificação para Sistemas de Diálogo}"  em que exploro e comparo diferentes técnicas de classificação de texto para o desenvolvimento do sistema em questão.

\section{Neurociência Cognitiva e Filosofia da Mente}
\textbf{\Large Descrição da Disciplina}\\

\textbf{Coordenação}\\
\hspace*{1.6em}Prof. Dr. Osvaldo Frota Pessoa Junior (FFLCH - USP)\\
\hspace*{1.6em}\textbf{Ministrante}\\
\hspace*{1.6em}Fabiana Mesquita de Carvalho\\

\textbf{Objetivos}\\
\hspace*{1.6em}Apresentar os princípios gerais da neurociência cognitiva, discutindo as conexões com questões da filosofia da mente.\\

\textbf{Forma de Avaliação}\\
\hspace*{1.6em}Sem Avaliação\\

\textbf{Relevância para a Pesquisa}

Esse curso de extensão foi fundamental para me aprofundar nos meus conhecimentos em neurociência. Nele, pude ter uma visão geral a respeito de diversos processos cognitivos, ter um melhor entendimento sobre o processamento dos cinco sentidos, entender processos como memória, emoções, linguagem, entre outros.\\



