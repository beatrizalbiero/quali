\documentclass[11pt,twoside,a4paper]{book}

% ---------------------------------------------------------------------------- %
% Pacotes 
\usepackage[T1]{fontenc}
\usepackage[english]{babel}
% \usepackage[latin1]{inputenc}
\usepackage[utf8x]{inputenc}
\usepackage[pdftex]{graphicx}           % usamos arquivos pdf/png como figuras
\usepackage{setspace}                   % espaçamento flexível
\usepackage{indentfirst}                % indentação do primeiro parágrafo
\usepackage{makeidx}                    % índice remissivo
\usepackage[nottoc]{tocbibind}          % acrescentamos a bibliografia/indice/conteudo no Table of Contents
\usepackage{courier}                    % usa o Adobe Courier no lugar de Computer Modern Typewriter
\usepackage{type1cm}                    % fontes realmente escaláveis
\usepackage{listings}                   % para formatar código-fonte (ex. em Java)
\usepackage{titletoc}
\usepackage{lmodern}
\usepackage{amsmath}
\usepackage{amsfonts}
\usepackage{booktabs}
\usepackage{bm}


%\usepackage[bf,small,compact]{titlesec} % cabeçalhos dos títulos: menores e compactos
\usepackage[fixlanguage]{babelbib}
\usepackage[font=small,format=plain,labelfont=bf,up,textfont=it,up]{caption}
\usepackage[usenames,svgnames,dvipsnames]{xcolor}
\usepackage[a4paper,top=2.54cm,bottom=2.0cm,left=2.0cm,right=2.54cm]{geometry} % margens
%\usepackage[pdftex,plainpages=false,pdfpagelabels,pagebackref,colorlinks=true,citecolor=black,linkcolor=black,urlcolor=black,filecolor=black,bookmarksopen=true]{hyperref} % links em preto
\usepackage[pdftex,plainpages=false,pdfpagelabels,pagebackref,colorlinks=true,citecolor=DarkGreen,linkcolor=NavyBlue,urlcolor=DarkRed,filecolor=green,bookmarksopen=true]{hyperref} % links coloridos
\usepackage[all]{hypcap}                    % soluciona o problema com o hyperref e capitulos
\usepackage[round,sort,nonamebreak]{natbib} % citação bibliográfica textual(plainnat-ime.bst)
\fontsize{60}{62}\usefont{OT1}{cmr}{m}{n}{\selectfont}

% ---------------------------------------------------------------------------- %
% Cabeçalhos similares ao TAOCP de Donald E. Knuth
\usepackage{fancyhdr}
\pagestyle{fancy}
\fancyhf{}

% new commands------------------------

\newcommand{\vect}[1]{\bm{#1}}
\newcommand{\myprime}[1]{{#1}^{\prime}}
\newcommand{\grad}[2]{\nabla_{#1} {#2}}
\newcommand{\dotp}[2]{{#1}^{\top}{#2}}
\newcommand{\dotpPright}[2]{{#1}^{\top}\left({#2}\right)}
\newcommand{\outerp}[2]{\left({#1}\right){#2}^{\top}}
\newcommand{\Jacobian}[2]{\frac{\partial #1}{\partial #2}}
\newcommand{\Vocab}{\mathbb{V}}
\newcommand{\corpus}{\mathbb{C}}
\newcommand{\A}{\mathcal{A}}
\DeclareMathOperator*{\argmin}{arg\,min}
 \DeclareMathOperator*{\argmax}{arg\,max}
\DeclareMathOperator{\E}{\mathbb{E}}

% new commands------------------------

\renewcommand{\chaptermark}[1]{\markboth{\MakeUppercase{#1}}{}}
\renewcommand{\sectionmark}[1]{\markright{\MakeUppercase{#1}}{}}
\renewcommand{\headrulewidth}{0pt}

% ---------------------------------------------------------------------------- %
\graphicspath{{./figuras/}}             % caminho das figuras (recomendável)
\frenchspacing                          % arruma o espaço: id est (i.e.) e exempli gratia (e.g.) 
\urlstyle{same}                         % URL com o mesmo estilo do texto e não mono-spaced
\makeindex                              % para o índice remissivo
\raggedbottom                           % para não permitir espaços extra no texto
\fontsize{60}{62}\usefont{OT1}{cmr}{m}{n}{\selectfont}
\cleardoublepage
\normalsize

% ---------------------------------------------------------------------------- %
% Opções de listing usados para o código fonte
% Ref: http://en.wikibooks.org/wiki/LaTeX/Packages/Listings
\lstset{ %
language=Python,                  % choose the language of the code
basicstyle=\footnotesize,       % the size of the fonts that are used for the code
numbers=left,                   % where to put the line-numbers
numberstyle=\footnotesize,      % the size of the fonts that are used for the line-numbers
stepnumber=1,                   % the step between two line-numbers. If it's 1 each line will be numbered
numbersep=5pt,                  % how far the line-numbers are from the code
showspaces=false,               % show spaces adding particular underscores
showstringspaces=false,         % underline spaces within strings
showtabs=false,                 % show tabs within strings adding particular underscores
frame=single,	                % adds a frame around the code
framerule=0.6pt,
tabsize=2,	                    % sets default tabsize to 2 spaces
captionpos=b,                   % sets the caption-position to bottom
breaklines=true,                % sets automatic line breaking
breakatwhitespace=false,        % sets if automatic breaks should only happen at whitespace
escapeinside={\%*}{*)},         % if you want to add a comment within your code
backgroundcolor=\color[rgb]{1.0,1.0,1.0}, % choose the background color.
rulecolor=\color[rgb]{0.8,0.8,0.8},
extendedchars=true,
xleftmargin=10pt,
xrightmargin=10pt,
framexleftmargin=10pt,
framexrightmargin=10pt
}

% Additional packages
\usepackage{lscape}

\usepackage{booktabs}
\usepackage{graphicx}
% \usepackage[table,xcdraw]{xcolor}
\usepackage{colortbl}

\usepackage{adjustbox}

\usepackage{amsmath}
\usepackage{amsthm}
\usepackage{amssymb}

\usepackage{cases}

\usepackage[ruled,vlined,linesnumbered]{algorithm2e}
\SetKwComment{Comment}{$\triangleright$\ }{}

\usepackage{tikz}
\usetikzlibrary{shapes,arrows,positioning,fit}
% Auxiliary styles =====================================================
\tikzstyle{textonly} = [draw=none, fill=none, text centered, font=\normalsize]
\tikzstyle{container} = [draw, fill=none, inner sep=0.3cm]

% Stochastic Computation Graphs ========================================

% nodes
\tikzstyle{input} = [draw=none, fill=none, minimum size = 15pt, text centered, font=\normalsize]
\tikzstyle{deterministic} = [rectangle, draw, minimum size = 20pt, text centered, font=\normalsize]
\tikzstyle{stochastic} = [circle, draw, minimum size = 20pt, text centered, font=\normalsize]

% edges
\tikzstyle{dedge}  = [draw, thick, >=stealth, ->]
\tikzstyle{pedge}  = [draw, thick, >=latex, ->]

% Deep Neural Nets =====================================================
\tikzstyle{inputvec} = [draw=none, fill=none, minimum size = 15pt, text centered, font=\normalsize]
\tikzstyle{outputvec} = [draw=none, fill=none, minimum size = 15pt, text centered, font=\normalsize]
\tikzstyle{unit} = [draw, circle]
\tikzstyle{layer} = [draw, rectangle, inner sep=0.2cm]
\tikzstyle{sublayer}  = [draw, dashed, >=stealth, ->]
\tikzstyle{activation}  = [draw, thick, >=stealth, ->]


\newtheorem{example}{Example}[section]
\newtheorem{definition}{Definition}[section]
\newtheorem{theorem}{Theorem}[section]
\newtheorem{corollary}{Corollary}[theorem]
\newtheorem{lemma}{Lemma}[theorem]
\newtheorem*{remark}{Remark}


% ---------------------------------------------------------------------------- %
% Corpo do texto
\begin{document}
\frontmatter 
% cabeçalho para as páginas das seções anteriores ao capítulo 1 (frontmatter)
\fancyhead[RO]{{\footnotesize\rightmark}\hspace{2em}\thepage}
\setcounter{tocdepth}{2}
\fancyhead[LE]{\thepage\hspace{2em}\footnotesize{\leftmark}}
\fancyhead[RE,LO]{}
\fancyhead[RO]{{\footnotesize\rightmark}\hspace{2em}\thepage}

\onehalfspacing  % espaçamento

% ---------------------------------------------------------------------------- %
% CAPA
% Nota: O título para as dissertações/teses do IME-USP devem caber em um 
% orifício de 10,7cm de largura x 6,0cm de altura que há na capa fornecida pela SPG.
\thispagestyle{empty}
\begin{center}
    \vspace*{2.3cm}
    \textbf{\Large{Adding semantic robustness to dialog agents}}\\
    
    \vspace*{1.2cm}
    \Large{Felipe de Souza Salvatore}
    
    \vskip 2cm
    \textsc{
    Research Project\\[-0.25cm]
    submitted to\\[-0.25cm]
    the\\[-0.25cm]
    Institute of Mathematics and Statistics\\[-0.25cm]
    of\\[-0.25cm]
    University of São Paulo\\[-0.25cm]
    for the\\[-0.25cm]
    Qualifying Examination\\[-0.25cm]
    for the\\[-0.25cm]
    Degree of Doctor of Philosophy}
    
    \vskip 1.5cm
    Program: Computer Science\\
    Supervisor: Professor Marcelo Finger, PhD\\
    Co-supervisor: Professor Roberto Hirata, PhD\\
    

   	\vskip 1cm
    \normalsize{During the development of this research project\\the author received financial support from CAPES}
    
    \vskip 0.5cm
    \normalsize{São Paulo, June, 2018}
\end{center}


\pagenumbering{roman}     % começamos a numerar


% ---------------------------------------------------------------------------- %
% Abstract
\chapter*{Abstract}
\noindent Salvatore, F. \textbf{Adding semantic robustness to dialog agents}. 
2018.
Research Project (PhD) - Institute of Mathematics and Statistics,
University of São Paulo, São Paulo, 2018.
\\
\\
Using the available dataset SICK (Sentence Involving Compositional Knowledge), we introduce a set of new question answering tasks \textbf{Entailment-QA} to measure how well a dialog system deals with abstract semantic knowledge. These tasks force the dialog agent to struggle with distinct notions that are at the intersection of text understanding and reasoning: boolean connectives, first-order quantifiers, synonymy/antinomy/hypernymy resolution, and paraphrase. Experimental results indicate that the current dialog systems present difficulties in solving these tasks.
\\

\noindent \textbf{Keywords:} Question answering, dialog systems, synthetic tasks, natural language processing.

% ---------------------------------------------------------------------------- %
% Sumário
\tableofcontents    % imprime o sumário

% ---------------------------------------------------------------------------- %
\chapter{Lista de Abreviações}
\begin{tabular}{ll}
\vspace{3mm}
\textbf{FFD} 		 & Feedforward\\ \vspace{3mm}
\textbf{RNN} 		 & Recurrent Neural Networks\\ \vspace{3mm}
\textbf{seq2seq} 	 & Sequence-to-Sequence\\ \vspace{3mm}

\end{tabular}


% ---------------------------------------------------------------------------- %
\chapter{Lista de Símbolos}
\begin{tabular}{ll}
\vspace{2mm}
\textbf{\#}   &Boundary\\ \vspace{2mm}
\textbf{<eos>}   &Token End of Sentence\\ \vspace{2mm}
$b$    &Bias\\ \vspace{2mm}
$\vect{c}$    &Matriz de Estados (Cell States)\\ \vspace{2mm}
$\vect{h}$    &Matriz de Estados\\ \vspace{2mm}
$\vect{x}$    &Vetor de Inputs\\ \vspace{2mm}
$\vect{\hat{y}}$    &Vetor de Outpus\\ \vspace{2mm}
$\vect{w}$    &Vetor\\ \vspace{2mm}
$\vect{W}$    &Matriz\\ \vspace{2mm}
%$\corpus$     &Corpus\\ \vspace{2mm}
%$\Vocab$      &Vocabulary \\ \vspace{2mm}
%$[\vect{a}; \vect{b}]$ &Concatenation of the vectors $\vect{a}$ and $\vect{b}$ \\ \vspace{2mm}
$\sigma$ & Função Sigmoid \\ \vspace{2mm}
$\tanh$ & Tangente Hiperbólica \\ \vspace{2mm}
\end{tabular}


% ---------------------------------------------------------------------------- %
% Listas de figuras e tabelas criadas automaticamente
\listoffigures            
% \listoftables            

% ---------------------------------------------------------------------------- %
% Capítulos do trabalho
\mainmatter

% cabeçalho para as páginas de todos os capítulos
\fancyhead[RE,LO]{\thesection}

% \singlespacing              % espaçamento simples
\onehalfspacing            % espaçamento um e meio


\chapter{Introdução}
\label{ch:01-introduction}

O processo de flexão verbal do tempo presente para o tempo passado na língua inglesa está certamente entre um dos temas de debate mais controversos entre as principais correntes teóricas no estudo da linguística (Rumelhart \& McClelland, 1986; Pinker \& Prince, 1988; Pinker, 1999). O cerne do debate está na exata caracterização dos mecanismos que possibilitam que um falante seja capaz relacionar um verbo no tempo presente à sua forma no tempo passado. 

O tempo passado do inglês é composto por uma variedade de famílias, ocorrendo não somente a distinção entre verbos regulares e irregulares mas também, a formação de grupos dentro do conjunto dos irregulares, compostos por verbos que compartilham do mesmo processo de flexão:

\begin{center}
blow – blew, grow – grew, know – knew, throw – threw\\
bear – bore, swear – swore, tear – tore, wear – wore\\
drink – drank, shrink – shrank, sink – sank, stink – stank \\
\end{center}

É possível pensar que o aprendizado da pertinência de um verbo a uma ou a outra família decorreria de uma memorização caso a caso. No entanto, experimentos realizados mostraram que, quando apresentados a verbos inventados, os indivíduos testados apresentaram tendências com relação à alocação dos verbos em classes, por exemplo, para o verbo artificial spling, a maioria das pessoas optou pela forma splang  ou splung (Bybee \& Moder, 1983). Esse exemplo contradiz a ideia de que os falantes poderiam estar apenas reproduzindo formas memorizadas e sugere que eles estejam ativamente identificando padrões, e mais: possuem uma intuição natural sobre a adequabilidade da alocação de um verbo a uma classe ou a outra.

Uma alternativa é pensar que as próprias unidades fonológicas das palavras possam fornecer pistas aos falantes que os permitam relacionar verbos com traços similares a uma mesma família. É a caracterização de tal processo de categorização que estabelece o debate entre as correntes teóricas racionalistas e empiristas.  

Para a teoria racionalista da Fonologia Gerativa de Chomsky e Halle (1968), os indivíduos seriam portadores de um dispositivo de aquisição de linguagem (Language Acquisition Device) responsável pela formulação e manipulação de estruturas fonológicas abstratas em um sistema intrincado de regras. De modo simplificado, a teoria propõe que o falante seja capaz de identificar e formular regras intuitivamente para dar conta do aprendizado das formas irregulares da língua. Um exemplo disso é a família dos verbos terminados em “-ind”.

\begin{center}
bind – bound, find – found, grind – ground, wind – wound
\end{center}

Vemos que, de modo simplificado, a regra em uso aqui é algo como:
%verificar essa formula fonetica
\begin{center}
ai $\rightarrow$ au / \_nd]+past
\end{center}

Uma estrutura como essa permitiria ao falante construir generalizações e, consequentemente, aprender com eficiência e rapidez. 

Apesar dos argumentos apresentados, a teoria racionalista foi confrontada com um forte questionamento. Tal questionamento apresenta-se com relação a este sistema de manipulação simbólica sugerido pela teoria racionalista, o qual os pesquisadores Rumelhart e McClelland (1986) intitularam de Regras Explícitas Inacessíveis (Explicit Inaccessible Rules). Rumelhart e McClelland argumentam que comportamentos de caráter regrado podem ser produzidos por mecanismos em que não existam representações explícitas das regras em uso. Ao invés disso, os pesquisadores sugerem que os mecanismos envolvidos no processo de flexão verbal possam ser construídos de tal forma que, a sua performance possa ser caracterizada por regras, mas que as regras em si não estejam representadas explicitamente em nenhuma parte do processo. Para sustentar essa ideia, Rumelhart e McClelland apresentaram um modelo computacional  de caráter empirista que foi fundamental para o surgimento de uma nova escola dentro das ciências cognitivas: o conexionismo.


\definecolor{blue}{RGB}{159, 192, 176}
\definecolor{green}{RGB}{160, 227, 127}
\definecolor{orange}{RGB}{243, 188, 125}
\definecolor{red}{RGB}{253, 123, 84}
\definecolor{nephritis}{RGB}{39, 174, 96}
\definecolor{emerald}{RGB}{46, 204, 113}
\definecolor{turquoise}{RGB}{39, 174, 96}
\definecolor{green-sea}{RGB}{22, 160, 133}
\definecolor{purple}{RGB}{181, 124, 215}
% Tikzstyles for Computation Graphs

% nodes
\tikzstyle{noop} = [circle, draw=none, fill=red, minimum size = 10pt]
\tikzstyle{op} = [circle, draw=red, line width=1.5pt, fill=red!70, text=black, text centered, font=\bf \normalsize, minimum size = 25pt]

\tikzstyle{opintense} = [circle, draw=red, line width=1.5pt, fill=red!150, text=black, text centered, font=\bf \normalsize, minimum size = 25pt]


%new style
\tikzstyle{gp} = [circle, draw=red, line width=4pt, text=black, text centered, font=\bf \normalsize, minimum size = 4.cm]

\tikzstyle{box} = [rectangle, draw=red, line width=1.5pt, fill=red!70, text=black, align=center, font=\bf \normalsize, minimum size = 45pt]

\tikzstyle{state} = [circle, draw=blue, line width=1.5pt, fill=blue!70, text=black, text centered, font=\bf \normalsize, minimum size = 25pt]

\tikzstyle{output} = [circle, draw=purple, line width=1.5pt, fill=purple!70, text=black, text centered, font=\bf \normalsize, minimum size = 25pt]


\tikzstyle{gradient} = [circle, draw=nephritis, line width=1.5pt, fill=nephritis!60, text=black, text centered, font=\bf \normalsize, minimum size = 25pt]
\tikzstyle{textonly} = [draw=none, fill=none, text centered, font=\bf \normalsize]
\tikzstyle{boxtextonly} = [draw=none, fill=none, align=center, font=\bf \normalsize]

\tikzstyle{normal} = [circle, draw=black, line width=1.0pt, fill=none, text=black, text centered, font=\bf \normalsize, minimum size = 20pt]


% edges
\tikzstyle{tedge}  = [draw, thick, >=latex, ->]
\tikzstyle{tedge_dashed}  = [draw, thick, >=latex, ->, dashed]
\tikzstyle{nedge}  = [draw, thick, >=latex]
\tikzstyle{nedge_dashed}  = [draw, thick, >=latex, dashed]


% namedscope
\tikzstyle{namedscope} = [circle, draw=orange, line width=1.5pt, fill=orange!60, align=center, inner sep=0pt]
\begin{figure}[ht!]
\centering

\scalebox{1.0}{
\begin{tikzpicture}[auto]

% operations =========
% phon features 1
\node[textonly] (1pho1) {int-vogal-int};

% Legenda
\node[textonly, above=10pt of 1pho1] (leg1) {Unidades de Input};


% FNN input
\node[normal, right=5pt of 1pho1] (x1) {};
\node[normal, below=25pt of x1] (x2) {};
\node[normal, below=25pt of x2] (x3) {};
\node[normal, below=25pt of x3] (x4) {};
\node[normal, below=25pt of x4] (x5) {};
\node[normal, below=25pt of x5] (x6) {};

% FNN output
\node[normal, right=45pt of x1] (y1) {};
\node[normal, right=45pt of x2] (y2) {};
\node[normal, right=45pt of x3] (y3) {};
\node[normal, right=45pt of x4] (y4) {};
\node[normal, right=45pt of x5] (y5) {};
\node[normal, right=45pt of x6] (y6) {};

% phon features 2
\node[textonly, right=5pt of y1] (2pho1) {int-vogal-int};
\node[textonly, above=10pt of 2pho1] (leg2) {Unidades de Output};
\node[textonly, left=25pt of x2] (1pho2) {anterior-nasal-posterior};
\node[textonly, right=25pt of y2] (2pho2) {anterior-nasal-posterior};
\node[textonly, left=25pt of x3] (3pho1) {...};
\node[textonly, right=25pt of y3] (1pho3) {...};
\node[textonly, left=25pt of x4] (4pho1) {nasal-cont-ocl};
\node[textonly, right=25pt of y4] (1pho4) {nasal-cont-ocl};
\node[textonly, left=25pt of x5] (5pho1) {médio-cont-baixa};
\node[textonly, right=25pt of y5] (1pho5) {médio-cont-baixa};
\node[textonly, left=25pt of x6] (6pho1) {vogal-fric-\#};
\node[textonly, right=25pt of y6] (1pho6) {vogal-fric-\#};
% edges FNN
\path[nedge] (x1) -- (y1);
\path[nedge] (x1) -- (y2);
\path[nedge] (x1) -- (y3);
\path[nedge] (x1) -- (y4);
\path[nedge] (x1) -- (y5);
\path[nedge] (x1) -- (y6);
\path[nedge] (x2) -- (y1);
\path[nedge] (x2) -- (y2);
\path[nedge] (x2) -- (y3);
\path[nedge] (x2) -- (y4);
\path[nedge] (x2) -- (y5);
\path[nedge] (x2) -- (y6);
\path[nedge] (x3) -- (y1);
\path[nedge] (x3) -- (y2);
\path[nedge] (x3) -- (y3);
\path[nedge] (x3) -- (y4);
\path[nedge] (x3) -- (y5);
\path[nedge] (x3) -- (y6);
\path[nedge] (x4) -- (y1);
\path[nedge] (x4) -- (y2);
\path[nedge] (x4) -- (y3);
\path[nedge] (x4) -- (y4);
\path[nedge] (x4) -- (y5);
\path[nedge] (x4) -- (y6);
\path[nedge] (x5) -- (y1);
\path[nedge] (x5) -- (y2);
\path[nedge] (x5) -- (y3);
\path[nedge] (x5) -- (y4);
\path[nedge] (x5) -- (y5);
\path[nedge] (x5) -- (y6);
\path[nedge] (x6) -- (y1);
\path[nedge] (x6) -- (y2);
\path[nedge] (x6) -- (y3);
\path[nedge] (x6) -- (y4);
\path[nedge] (x6) -- (y5);
\path[nedge] (x6) -- (y6);


\end{tikzpicture}
}\caption{Esquema da rede neural utilizada pelos pesquisadores Rumelhart e McClelland} 
\label{fig:esquemafdd}
\end{figure}


O modelo desenvolvido foi criado por analogia à estrutura em que se relacionam os neurônios no cérebro, por isso, recebeu o nome de rede neural artificial (artificial neural network). Ele é composto basicamente por uma rede artificial de nódulos interconectados paralelamente (Fig. \ref{fig:esquemafdd}).




A primeira camada de nódulos é responsável por receber os dados de entrada (input), que são os dados referentes aos traços fonológicos que caracterizam os sons de um verbo no tempo presente. A segunda camada é uma camada de resposta (output) que deve tentar retornar dados referentes aos traços que caracterizam os sons do mesmo verbo fornecido no input, porém no tempo passado. Concluída esta etapa, os dados de saída obtidos deverão ser então comparados à forma correta do verbo no tempo passado, uma espécie de gabarito. A função das conexões entre as camadas é fortalecer (ou enfraquecer) as relações entre as camadas de input e output de acordo com as comparações realizadas entre a camada de output e o gabarito. É importante notar que, a priori, a rede não possui qualquer tipo de informação para seu funcionamento, essa aprendizagem irá decorrer ao longo de múltiplas iterações.

O modelo de Rumelhart e McClelland apresentou ótimos resultados na tarefa de prever as formas verbais esperadas para o past simple, conseguindo identificar associações corretamente entre todos os 420 verbos em que foi treinado. Além disso, teve um desempenho satisfatório ao ser apresentado a 86 novos verbos que não fizeram parte do treinamento, obtendo uma taxa de acerto de 92\% para verbos irregulares e 84\% para verbos irregulares (91\% de acerto para todos os verbos no total). O modelo serviu, portanto, para corroborar o argumento de que é possível realizar essa tarefa eficientemente dispensando o uso de regras explícitas. Além disso, o processo de aprendizagem do modelo computacional exibiu uma performance muito interessante, reproduzindo resultados similares a comportamentos observáveis em crianças durante a fase de aquisição: a Curva de Desenvolvimento em U (U-shaped Development, Marcus et al. 1992). Na fase inicial do processo, o modelo foi exposto a uma quantidade pequena de verbos de alta frequência na língua inglesa, como: come, get, give, look, take, go, have, live e feel. A performance do modelo foi compatível com o primeiro estágio da curva, ou seja, para uma pequena quantidade de verbos, foi capaz de identificar corretamente a forma correspondente no passado simples. Em um segundo momento, o modelo foi exposto a uma quantidade muito maior de verbos. Esse estágio é interessante porque fica evidente que o modelo está passando por um processo de regularização sistemática dos verbos. O modelo produziu resultados como: breaked, comed, gived; e também combinações entre padrões regulares e irregulares (ex. gaved),  compatível com o estágio intermediário do processo de aprendizagem. Após uma série de ensaios, o modelo foi finalmente capaz de responder corretamente a uma grande quantidade de verbos, assim como no último estágio do processo da aprendizagem natural. 

Embora o modelo conexionista de Rumelhart e McClelland tenha apresentado um desempenho muito interessante, não ficou livre de críticas. Pinker e Prince (1988) dão continuidade ao debate ao apontar uma série de questões pertinentes que a proposta empirista falhou em explicar.  Pelo fato de ser meramente um mecanismo associativo entre traços fonológicos, o modelo acaba dependendo profundamente dos padrões encontrados entre os traços fonológicos das palavras fornecidas no treinamento. Isso significa que o modelo é incapaz de responder de maneira eficiente a verbos com traços fonológicos que não passaram pelo treinamento. Pinker (1999) destaca a vulnerabilidade do modelo diante da tarefa de prever a forma no passado para alguns verbos costumeiros na língua inglesa, mas que dispõem de uma sonoridade razoavelmente incomum. Quando exposto aos verbos jump, pump, warm,  trail e glare, o modelo não gerou nenhuma resposta. Além disso, apresentou alguns resultados completamente distorcidos, como: squat – squakt, tour – toureder e mail – membled; associações inaceitáveis para qualquer falante nativo. A solução que Pinker e Prince (1988) encontraram para a formulação de uma nova teoria linguística mais contundente, foi justamente propor uma teoria híbrida em que Chomsky e Halle estariam basicamente corretos quanto ao processo de flexão regular e Rumelhart e McClelland corretos quanto ao processo de flexão irregular. Pinker e Prince propõem que as formas regulares sejam computadas a partir de um mecanismo que deve abstrair o stem do verbo e combiná-lo ao sufixo –ed.  Tal mecanismo pode ser aplicado a qualquer palavra, em um processo independente da memória. As formas irregulares passam por um processo diferente. De fato, verbos irregulares precisam passar por um processo de memorização, mas essa memorização ocorre de maneira associativa, havendo não somente a associação entre uma palavra a outra mas também entre as propriedades (traços fonológicos, rima, stem, núcleo, etc.) de uma palavra e de outra, parecido com o que foi proposto por Rumelhart e McClelland.
\section{Motivação}
\label{sec:motivation}

Em comparação com o inglês, o português brasileiro apresenta um sistema verbal flexional mais complexo. Primeiramente, é importante notar que o paradigma conjugacional do português brasileiro apresenta uma maior distinção flexional quanto a pessoas e número, enquanto que o inglês é mais limitado nesse aspecto. Quanto às formas regulares, o português dispõe de três conjugações diferentes determinadas pela vogal temática, enquanto que o inglês, possui apenas a conjugação regular de passado (o sufixo –ed). Por último, é importante destacar que o sistema verbal do português é repleto de irregularidades em todos os tempos verbais, enquanto que o inglês que apresenta irregularidades apenas no past simple e past participle (Wuerges, 2014).

Um aprendiz da linguagem no sistema do português brasileiro é desafiado a superar uma série de obstáculos.  Uma parte do processo é justamente perceber a relação entre a vogal temática e as possíveis conjugações verbais regulares. Nesse processo não é incomum observarmos o surgimento de trocas de conjugação como: eu boti*, eu comei*, eu aprendei*, eu janti* (Wuerges, 2014). Outra dificuldade é ter de lidar com o fato de que os verbos irregulares no português apresentam, em pelo menos uma forma verbal de seu paradigma, alterações no radical e/ou na sua desinência. Isto fica evidente quando observamos a enunciação de formas como: “eu consego*” ou “eu podo*” (poder). É interessante também notar enunciações criativas para verbos de natureza um pouco mais complicada, como o verbo por:  puso* (eu), ponhei* (eu) (Wuerges, 2014).

Uma análise mais profunda sobre a disposição das irregularidades presentes no português brasileiro (levando em consideração apenas 1a pessoa do singular (tempo presente - modo indicativo) nos permite observar algumas regularidades (padrões particulares) dentre os verbos irregulares:\\

\begin{center}

Bobear – Bobeio, Bloquear – Bloqueio, Chatear – Chateio, Clarear – Clareio, Golpear – Golpeio;\\

Agredir – Agrido, Conseguir – Consigo, Inserir – Insiro, Perseguir – Persigo, Preferir – Prefiro, Proferir – Profiro, Repetir – Repito, Servir –  Sirvo, Vestir – Visto;\\

Cobrir – Cubro, Dormir – Durmo, Engolir – Engulo;\\

 Al[e]gar – Al[ε]go, C[e]gar – C[ε]go, Compl[e]tar – Compl[ε]to,  Col[e]tar – Col[ε]to, Entr[e]gar – Entr[ε]go, Pr[e]gar – Pr[ε]go;\\

Ad[o]rar – Ad[\textopeno]ro, Ad[o]tar – Ad[\textopeno]to, B[o]tar – B[\textopeno]to, C[o]lar – C[\textopeno]lo, F[o]car – F[\textopeno]co, M[o]rar – M[\textopeno]ro, S[o]ltar – S[\textopeno]lto, S[o]lar – S[\textopeno]lo, T[o]car – T[\textopeno]co, M[o]strar – M[\textopeno]stro;\\

Mentir - Minto, Sentir - Sintu;

\end{center}

Os padrões observados a partir da exposição de algumas classes irregulares, permitem, assim como no inglês, a proposição de fórmulas, ou regras fonológicas, que explicam as flexões realizadas em cada classe. É possível notar, por exemplo, que um verbo com a vogal temática \textit{'e'}, quando inserido no contexto de uma terceira conjugação resulta na inflexão deste fonema para o fonema \textit{'i'}:

% Inserir regrinha formal

As previsibilidades encontradas sugerem não somente a possibilidade de elaboração das regras fonológicas, como também a possibilidade do desenvolvimento de redes capazes de capturar tais dependências. A dificuldade, no entanto, se apresenta quando observam-se os comprimentos médios dos verbos em cada língua. No inglês, o comprimento médio dos --- verbos mais comuns da língua é -- %colocar referencia do calculo
. Quando se leva em consideração a transcrição fonética dos mesmos, é ainda menor. Verbos relativamente compridos como \textit{bought, thought, knew} são reduzidos a sequências curtas de sons: bot, tot, niu. %arrumar essas transcricoes
No português, o comprimento médio dos --- verbos mais comuns é de ---. E levando em consideração a transcrição fonética dos mesmos, cai para --. %( ou nao muda, sei la)
Tal diferença apresenta-se como um desafio, pois torna mais complexa a tarefa de assimilação de dependências de longa distância.

% ------------------------------------------------------------------------
\section{Objetivos}
\label{sec:objectives}

O principal objetivo do projeto será construir um modelo de redes neurais artificiais e avaliá-lo quanto a sua capacidade em detectar as relações fonológicas existentes na conjugação de verbos irregulares do português brasileiro, tendo como parâmetro comparativo o resultado de uma avaliação psicolinguística em falantes nativos.
Para tanto, foram definidos os objetivos específicos:

\begin{itemize}
\item Construção de modelos preditivos computacionais (redes neurais artificiais) capazes de detectar os padrões morfofonológicos existentes na conjugação de verbos irregulares do português brasileiro.

\item Análise profunda quanto às regularidades e irregularidades verbais presentes no português brasileiro;

\item Desenvolvimento e aplicação de uma avaliação psicolinguística a fim de testar a intuição linguística de falantes nativos diante da tarefa de conjugar pseudoverbos que incorporam os paradigmas morfofonológicos das famílias de verbos irregulares apresentadas.

\end{itemize}

% ------------------------------------------------------------------------
\section{Organização}
\label{sec:organization}

O capítulo 2 irá explorar a base teórica que foi utilizada para o desenvolvimento desta pesquisa. O capítulo 3 apresentará resultados preliminares encontrados a partir das construções das redes. O Capítulo 4 discutirá os próximos passos da pesquisa.
\begin{enumerate}
\item 
\end{enumerate}

\chapter{Background}
\label{ch:02-background}

We can make human language manageable to computers by using a sort of different methods and techniques. This is the bread and butter of any NLP researcher. Here, following the NLP community, we will focus only in the techniques derived from the machine learning field. In this chapter, we will present some abstract machine learning models follow by some applications in NLP. 

\section{Machine Learning}

\textit{Machine learning} is the branch of computer science that deals with programs that can improve with experience (i. e., learn). Machine learning is divided into three main subareas: \textit{supervised learning}, \textit{reinforcement learning} and \textit{unsupervised learning}.

We will work with supervised learning only. In this setting, we assume that the process of interest is defined by an unknown function $g:X\rightarrow Y$. We try to approximate $g$ by changing the parameters of a function $f$ through an optimization process. The optimization is done by defining an \textit{error function} that evaluates how well $f$ approximates $g$ in the part of $g$ that we have access: the training data $D = \{(\vect{x}^{(1)}, \vect{y}^{(1)}), \dots ,(\vect{x}^{(N)}, \vect{y}^{(N)})\}$ (where $g(\vect{x}^{(i)})=\vect{y}^{(i)}$).


$f$ can be a function from any family of models. One family that is having a lot of success for language tasks is the \textit{neural network} family.

\section{Neural Networks}

A neural network is a non-linear function $f(\vect{x}; \vect{\theta})$. It is defined by a collection of parameters $\vect{\theta}$ and a collection of non-linear transformations. It is usual to represent $f$ as a compositions of functions:

\begin{align}
f(\vect{x}; \vect{\theta}) &= f^{(2)}(f^{(1)}(\vect{x}; \vect{W}_1, \vect{b}_1); \vect{W}_2, \vect{b}_2)\\
&= softmax(\vect{W}_2 (\sigma(\vect{W}_1\vect{x} + \vect{b}_1)) + \vect{b}_2)
\end{align}


The output of these intermediary functions are referred as \textit{layers}. So in the example above, $\vect{x}$ (the output of the identity function) is the \textit{input layer}, $f^{(1)}(\vect{x}; \vect{W}_1, \vect{b}_1)$ is the \textit{hidden layer} and $f^{(2)}(f^{(1)}(\vect{x}; \vect{W}_1, \vect{b}_1); \vect{W}_2, \vect{b}_2)$ is the \textit{output layer}. Since each layer is a vector, we normally speak about the \textit{dimension} of a layer. For historical reasons we also say that each entry on a layer is a \textit{node} or a \textit{neuron}.  Models with a large number of hidden layers are called \textit{deep models}, for this reason the name \textit{deep learning} is used.  

\par A neural network is a function approximator: it can approximate any Borel measurable function from one finite dimensional space to another with any desired nonzero amount of error. This theoretical result is know as the \textit{universal approximation theorem} \cite{Cybenko}. Without entering in the theoretical concepts, it suffice to note that the family of Borel mensurable functions include all continuous functions on a closed and bounded subset of $\mathbb{R}^n$.



Different deep learning architectures are used in NLP: \textit{convolutional architectures} have a good performance in tasks were it is required to find a linguistic indicator regardless of its position (e.g., document classification, short-text categorization, sentiment classification, etc); high quality word embeddings can be achieved with models that are a kind of \textit{feedforward neural network} \cite{Mikolov23}. But for a variety of works in natural language we want to capture regularities and similarities in a text structure. That is way \textit{recurrent} and \textit{recursive} models have been widely used in the field. Here we are focused on generative models and since recurrent models have been producing very strong results for language modeling \cite{goldberg15}, we will concentrate on them.


\section{Recurrent Neural Network}
\label{sec:RNN}


\textit{Recurrent neural network} (RNN) is a family of neural network specialized in sequential data $\vect{x}^{(1)}, \dots, \vect{x}^{(\tau)}$. As a neural network, a RNN is a parametrized function that we use to approximate one hidden function from the data. What makes RNN unique is a recurrent definition of one of its hidden layer:

\begin{equation}
\vect{h}^{(t)} = g(\vect{h}^{(t-1)}, \vect{x}^{(t)}; \vect{\theta})
\end{equation}

$\vect{h}^{(t)}$ is called \textit{state}, \textit{hidden state}, or \textit{cell}.


\par This recurrent equation can be unfolded for a finite number of steps $\tau$. For example, when $\tau =3$:

\begin{align}
\vect{h}^{(3)}& = g(\vect{h}^{(2)}, \vect{x}^{(3)}; \vect{\theta})\\
 & = g(g(\vect{h}^{(1)}, \vect{x}^{(2)}; \vect{\theta}), \vect{x}^{(3)}; \vect{\theta})\\
 & = g(g(g(\vect{h}^{(0)}, \vect{x}^{(1)}; \vect{\theta}), \vect{x}^{(2)}; \vect{\theta}), \vect{x}^{(3)}; \vect{\theta})\\
\end{align}

Using a concrete example consider the following classification model define by the equations:

\begin{equation}
f(\vect{x}^{(t)}, \vect{h}^{(t-1)}; \vect{V}, \vect{W}, \vect{U}, \vect{c}, \vect{b}) = \vect{\hat{y}}^{(t)}
\end{equation}
 \vspace{0.2cm}
\begin{equation}
\vect{\hat{y}}^{(t)} = softmax(\vect{V} \vect{h}^{(t)} + \vect{c})
\end{equation}
\vspace{0.2cm}
 \begin{equation}
\vect{h}^{(t)} = g(\vect{h}^{(t-1)}, \vect{x}^{(t)}; \vect{W},\vect{U}, \vect{b})
\end{equation}
\vspace{0.2cm}
\begin{equation}
\vect{h}^{(t)} = \sigma(\vect{W} \vect{h}^{(t-1)} + \vect{U} \vect{x}^{(t)} + \vect{b})
\end{equation}


This kind of model can create an output $\vect{\hat{y}}^{(t)}$ at each time $t$, or the model can produce a single output $\vect{\hat{y}}$ after processing an entire input sequence. This choice depends on the learning problem that is being modeled.


With the model's prediction at hand, we can use a loss function (like cross entropy for the classification problem) and apply the back-propagation algorithm to optimize the model. These models look complex, but it quite straightforward to compute the gradients \cite[p.~374]{DeepLearningbook}.

Although this kind of deep learning model is very useful, it presents a severe flaw. When computing the gradients there is a lot of repeated matrix multiplication using the recurrent weight matrix (in the example above, the matrix $\vect{W}$). Depending on some configurations of this matrix \textit{the gradients may vanish or explode exponentially with respect to the number of time steps}.

Thus, handing long-term dependencies became a problem when using RNNs. Different solutions were proposed, the most effective results came from some modifications of this model. We will present the two most famous modifications: \textit{the gated recurrent unit} and \textit{the long short-term memory}. 


\section{Gated Recurrent Unit}
\label{sec:GRU}

To capture long-term dependencies on a RNN  the authors of the paper \cite{ChungGCB14}  proposed a new architecture called \textit{gated recurrent unit} (GRU). This model was constructed to make each hidden state  $\vect{h}^{(t)}$ to adaptively capture dependencies of different time steps. It work as follows, at each step $t$ one candidate for hidden state is formed:

\begin{equation}
\vect{\widetilde{h}}^{(t)} = tahn(\vect{W} (\vect{h}^{(t-1)} \odot  \vect{r}^{(t)}) + \vect{U} \vect{x}^{(t)} + \vect{b})
\end{equation}

where $\vect{r}^{(t)}$ is a vector with values in $[0, 1]$ called a \textit{reset gate}, i.e.,  a vector that at each entry outputs the probability of reseting the  corresponding entry in the previous hidden state $\vect{h}^{(t-1)}$. Together with $\vect{r}^{(t)}$ we define an \textit{update gate}, $\vect{u}^{(t)}$. It is also a vector with values in $[0, 1]$. Intuitively we can say that this vector decides how much on each dimension we will use the candidate update. Both $\vect{r}^{(t)}$ and $\vect{u}^{(t)}$ are defined by $\vect{h}^{(t-1)}$ and $\vect{x}^{(t)}$; they also have specific parameters:

\begin{equation}
\vect{r}^{(t)} = \sigma(\vect{W}_{r} \vect{h}^{(t-1)} + \vect{U}_{r} \vect{x}^{(t)} + \vect{b}_{r})
\end{equation}


\begin{equation}
\vect{u}^{(t)} = \sigma(\vect{W}_{u} \vect{h}^{(t-1)} + \vect{U}_{u} \vect{x}^{(t)} + \vect{b}_{u})
\end{equation}

At the end the new hidden state $\vect{h}^{(t)}$ is defined by the recurrence:

\begin{equation}
\vect{h}^{(t)} = \vect{u}^{(t)} \odot \vect{\widetilde{h}}^{(t)} + (1 - \vect{u}^{(t)}) \odot \vect{h}^{(t-1)} 
\end{equation}

Note that the new hidden state combines the candidate hidden state $\vect{\widetilde{h}}^{(t)}$ with the past hidden state $\vect{h}^{(t-1)}$ using both $\vect{r}^{(t)}$ and $\vect{u}^{(t)}$ to adaptively copy and forget information.

\section{Long Short-Term Memory}
\label{sec:LSTM}

\textit{Long short-term memory} (LSTM) is one of the most applied versions of the RNN family of models. Historically it was developed before the GRU model, but conceptually we can think in the LSTM as an expansion of the model presented in the last session. Because of notation differences they can look different. LSTM is also based on parametrized gates; in this case three: the \textit{forget gate}, $\vect{f}^{(t)}$, the \textit{input gate}, $\vect{i}^{(t)}$, and the \textit{output gate}, $\vect{o}^{(t)}$. The gates are defined only by $\vect{h}^{(t-1)}$ and $\vect{x}^{(t)}$ with specific parameters:


\begin{equation}
\vect{f}^{(t)} = \sigma(\vect{W}_{f} \vect{h}^{(t-1)} + \vect{U}_{f} \vect{x}^{(t)} + \vect{b}_{f})
\end{equation}

\begin{equation}
\vect{i}^{(t)} = \sigma(\vect{W}_{i} \vect{h}^{(t-1)} + \vect{U}_{i} \vect{x}^{(t)} + \vect{b}_{i})
\end{equation}

\begin{equation}
\vect{o}^{(t)} = \sigma(\vect{W}_{o} \vect{h}^{(t-1)} + \vect{U}_{o} \vect{x}^{(t)} + \vect{b}_{o})
\end{equation}

Intuitively $\vect{f}^{(t)}$ should control how much informative will be discarded, $\vect{i}^{(t)}$ controls how much information will be updated, and $\vect{o}^{(t)}$ controls how munch each component should be outputted. A candidate cell, $\tilde{\vect{c}}^{(t)}$ is formed:

\begin{equation}
\tilde{\vect{c}}^{(t)} = tahn(\vect{W} \vect{h}^{(t-1)} + \vect{U} \vect{x}^{(t)} + \vect{b})
\end{equation}

And a new cell $\vect{c}^{(t)}$ is formed by forgetting some information of the previous cell $\tilde{\vect{c}}^{(t-1)}$ and by adding new values from $\tilde{\vect{c}}^{(t)}$ (scaled by the input gate)

\begin{equation}
\vect{c}^{(t)} = \vect{f}^{(t)} \odot \vect{c}^{(t-1)} + \vect{i}^{(t)} \odot \tilde{\vect{c}}^{(t)}
\end{equation}

The new hidden state, $\vect{h}^{(t)}$, is formed by filtering $\vect{c}^{(t)}$:

\begin{equation}
\vect{h}^{(t)} = \vect{o}^{(t)} \odot tanh(\vect{c}^{(t)})
\end{equation}

Until now we have presented general deep learning theory, now we will focus on the specificities of these models applied to natural language problems.


\section{Language model}

We call \textit{language model} a probability distribution over sequences of tokens in a natural language.

\[
P(x_1,x_2,x_3,x_4) = p
\]

This model is used for different NLP tasks such as speech recognition, machine translation, text auto-completion, spell correction, question answering, summarization and many others.

The classical approach to a language model was to use the chain rule of probability and a Markovian assumption, i.e., for a specific $n$ we assume that:

\begin{equation}
P(x_1, \dots, x_T) = \prod_{t=1}^{T} P(x_t \vert x_1, \dots, x_{t-1}) = \prod_{t=1}^{T} P(x_{t} \vert x_{t - (n+1)}, \dots, x_{t-1})
\end{equation} 


This gave raise to models based on $n$-gram statistics. The choice of $n$ yields different models; for example, the 
\textit{unigram} language model ($n=1$) is defined as: 
\begin{equation}
P_{uni}(x_1, x_2, x_3, x_4) = P(x_1)P(x_2)P(x_3)P(x_4)
\end{equation}

where $P(x_i) = count(x_i)$ and $count$ is a function that counts tokens occurrence in a corpus.\\

Similarly the \textit{bigram} language model ($n=2$) is defined as: 
\begin{equation}
P_{bi}(x_1,x_2,x_3,x_4) = P(x_1)P(x_2\vert x_1)P(x_3\vert x_2)P(x_4\vert x_3)
\end{equation} 
where
\begin{equation}
P(x_i\vert x_j) = \frac{count(x_i, x_j)}{count(x_j)}
\end{equation} 

With these basic statistics we can already define useful language models. It is observed that higher $n$-grams yields better performance. This comes with a price though, higher $n$-grams requires great amounts of memory \cite{Heafield}. For this motive $n$-grams based language models that are trained on large corpora uses at most $5$-grams. 

Since \cite{Mikolov11} the landscape has change, instead of using one approach that is specific for the language domain, we can use a general model for sequential data prediction, a RNN. The RNN's ability to deal with unrestricted size sequence input permits to abandon the $n$-gram model's restricted context assumption.

To understand the language model task as a machine learning task, some details should be clear. 

First, the learning task is to estimate the probability distribution 

\begin{equation}
\label{languagedistri}
P(x_{n} = \text{word}_{j^{*}} | x_{1}, \dots ,x_{n-1})
\end{equation}

for any $(n-1)$-sequence of words $x_{1}, \dots ,x_{n-1}$.


Second, the function $f$ being used to approximate \ref{languagedistri} is trained on a corpus in the following way: each word $x_t$ of the corpus will be used as input to $f$ (in the form of an one-hot vector), and the immediate subsequent word, say $x_{t+1}$, will be used as a target. The training is done by minimizing the cross entropy loss between the model's output and the probability distribution given by the target.

One example is in order. Suppose our corpus $\corpus$ is compose only by the lines below:

\begin{quote}
Yes, here we go again, give you more, nothing lesser\\
Back on the mic is the anti-depressor\\
Ad-Rock, the pressure, yes, we need this\\
The best is yet to come, and yes, believe this\\
\end{quote}


From this corpus we can construct a vocabulary list $\Vocab$ as follows: after preprocessing the text we create a list of the most frequent words (in this case we can take all words) with the size $V$ (here $V=27$). Hence, we can treat each word in the text either by an index referring to the word position on $\Vocab$ or as a one-hot vector that codifies this index (e.g., "the" would be identified with $0$, "yes" with $1$, "we" with $2$, and so on).

Then, the dataset is the collection of words 
\[
D = \{(<eos>, \text{Yes}), (\text{Yes}, \text{here}), (\text{here}, \text{we}),\dots,(\text{believe}, \text{this}), (\text{this}, <eos>)\}
\]
where $<eos>$ is the "end of sentence" token (also a member of $\Vocab$).

A simple recurrent language model $f(\vect{x}^{(t)}, \vect{\theta})$ is defined by the following equations:

\begin{equation}
\vect{e}^{(t)} = \vect{E}\vect{x}^{(t)}
\end{equation}
\vspace{0.2cm}
\begin{equation}
\vect{h}^{(t)} = \sigma(\vect{W}\vect{h}^{(t-1)}+ \vect{U}\vect{e}^{(t)}+ \vect{b})
\end{equation}
\vspace{0.2cm}
\begin{equation}
f(\vect{x}^{(t)}, \vect{\theta}) = \vect{\hat{y}}^{(t)} = softmax(\vect{V}\vect{h}^{(t)} + \vect{c})
\end{equation}

where $\vect{E} \in \mathbb{R}^{d,V}$ is the matrix of word embeddings, $\vect{x}^{(t)} \in \mathbb{R}^{V}$ is one-hot word vector at time step $t$, $\vect{y}^{(t)} \in \mathbb{R}^{V}$ is the ground truth at time step $t$ (also a one-hot word vector) and $d$ is the size of the word embeddings.

For each word $x_t$ let $j_t$ be the index of the subsequent word, so at each time $t$ the point-wise loss is:

\begin{align}
\label{lossCE}
L^{(t)}(\vect{\theta}) &= CrossEntropy(\vect{y}^{(t)},\vect{\hat{y}}^{(t)})\\
    &= - \log({\vect{\hat{y}}^{(t)}}_{j_t})\\
        &= - \log P(x_{t+1} = \text{word}_{j_t}|x_{1}, \dots, x_{t})\\
        &= - \log P(x_{t+1}|x_{1}, \dots, x_{t})
\end{align}

The loss $L$ is the mean of all point-wise losses

\begin{equation}
L(\vect{\theta})=\frac{1}{T}\sum_{t=1}^{T}L^{(t)}(\vect{\theta})
\end{equation}

With the loss function defined, we apply some optimization algorithm like \textit{stochastic gradient descent} to choose the optimal parameters for the language model:

\begin{equation}
\vect{\theta}^{*} = \argmin_{\vect{\theta}} L(\vect{\theta})
\end{equation}

Because of the historical connections with information theory, the \textit{perplexity} ($PP$) metric is often used to evaluate a language model. This metric can be thought as the weighted average branching factor of a language.

Given $\corpus = x_1, x_2, \dots, x_T$, we define the perplexity of $\corpus$ ($PP(\corpus)$) as:

\begin{align}
PP(\corpus) &= P(x_1, x_2, \dots, x_T)^{-\frac{1}{T}}\\
      &= \sqrt[T]{\frac{1}{P(x_1, x_2, \dots, x_T)}}\\
      &= \sqrt[T]{\prod_{i=1}^{T}\frac{1}{P(x_i \vert x_1,\dots, x_{i-1})}}
\end{align}

Using \ref{lossCE} we can relate cross entropy loss and perplexity:

\begin{align}
        L(\vect{\theta}) &=\frac{1}{T} \sum_{t=1}^{T} L^{(t)}(\vect{\theta})\\
          &=\frac{1}{T} \sum_{t=1}^{T} - \log P(x_{t+1}|x_{1}, \dots, x_{t})\\
          &=\frac{1}{T} \sum_{t=1}^{T} \log ((\frac{1}{P(x_{t+1}|x_{1}, \dots, x_{t})})\\
          &= \log\left( \sqrt[T]{\prod_{i=1}^{T}\frac{1}{P(x_i \vert x_1,\dots, x_{i-1})}} \right)\\
          &= \log(PP(\corpus))
\end{align}

Hence,

\begin{equation}
2^{L(\vect{\theta})} = PP(\corpus)
\end{equation}

Thus, by finding the parameters that minimize the cross entropy error we are also minimizing the perplexity of the language model. 

\section{Sequence-to-Sequence}
\label{sec:Seq2seq}

There is one powerful application of RNN based language model. The authors from \cite{Sustskever} used two RNNs to create an end-to-end translation model that is now know as \textit{the encoder-decoder} or \textit{the sequence-to-sequence} (seq2seq) architecture. This architecture is define as follows: let $\vect{x}^{(1)}, \dots, \vect{x}^{(n)}$ be a source sentence in the one-hot representation,  let $\vect{y}^{(1)}, \dots, \vect{y}^{(m)}$ be the target sentence also in the one-hot format. $f_{enc}$ (the \textit{encoder}) is a RNN with the sole purpose of creating a vector representation of input language's sequences. $f_{dec}$ (the \textit{decoder}) is a language model for the target language. These models are trained together mapping source sentences to target sentences.

For example, suppose the training pair $(\vect{x}^{(1)}, \dots, \vect{x}^{(n)}, \vect{y}^{(1)}, \dots, \vect{y}^{(m)})$ is ("Nas tardes de fazenda há muito azul demais", "In the farm’s afternoons there is too much blue"). Here we want the model to translate one Portuguese sentence to an English one. We first encode the Portuguese sentence in the vector $\vect{s}$, i.e.,

\begin{equation}
\vect{s} = f_{enc}(\vect{x}^{(n)}, \vect{h}^{(n-1)})
\end{equation}

Then using the control English sentence as a target, at each time $t$ we compute the cross entropy error between the decoder prediction $f_{dec}(\vect{y}^{(t)}, \vect{\tilde{h}}^{(t-1)})$ and the target $\vect{y}^{(t+1)}$ ($\vect{y}^{(0)}$ is the $<eos>$ token). The decoder uses the vector representation of the source sentence $\vect{s}$ as an initial hidden state (i.e., $\vect{\tilde{h}}^{(0)} = \vect{s}$). The goal of this model is to to approximate the probability distribution over the tokens from the target language given the sentence of the source language, i.e.,

\begin{equation}
\vect{\tilde{h}}^{(t)} = f_{dec}(\vect{y}^{(t)}, \vect{\tilde{h}}^{(t-1)})
\end{equation}

\begin{equation}
\label{decpred}
p(y_t | y_1, \dots, y_{t-1}, x_1, \dots, x_{n}) = softmax(\vect{W}_{s}  \vect{\tilde{h}}^{(t)} + \vect{b}_s)
\end{equation}

One limitation of this architecture is that the source sentence, in some cases, has more features than the decoder embedding $\vect{s}$ can properly summarize. To address that some attention mechanisms were introduced.

\section{Attention}
\label{sec:Attention}

The attention-based models are models built on top of the seq2seq architecture. The encoding part continues the same as before, but now at each time $t$ a context vector $\vect{c}^{(t)}$ is defined to capture relevant source-side information to help the prediction of the current target word $\vect{y}^{(t)}$. Once $\vect{c}^{(t)}$ is constructed the attention hidden state is defined as:   

\begin{equation}
\vect{\tilde{h}}^{(t)} = tahn(\vect{W}_c[\vect{c}^{(t)};\vect{h}^{(t)}])
\end{equation}

With the attention hidden state defined, the model's prediction is the same as the one defined in \ref{decpred}.

The core of this technique is the definition of $\vect{c}^{(t)}$. There are different strategies available, here we will focus only on one: \textit{global attention}.

Let $\vect{a} \in \mathbb{R}^{m,n}$. We will use this matrix as an alignment matrix, i.e., at the end of the training $\vect{a}_{ts}$ should reflect the probability of the source representation $\vect{h}^{(s)}$ be relevant for the output $\hat{y}^{(t)}$. We define $\vect{a}_{ts}$ as


\begin{equation}
\vect{a}_{ts} = \frac{exp(score(\vect{\tilde{h}}_t,\vect{h}_s))}{\sum_j exp(score(\vect{\tilde{h}}_t,\vect{h}_j))}
\end{equation}

Where $score$ is a content-based function that can have different implementations: 

\begin{equation}
score(\vect{\tilde{h}}_t,\vect{h}_s) = \begin{cases}
\vect{\tilde{h}}_t ^{\top}\vect{h}_s\\
\vect{\tilde{h}}_t ^{\top}\vect{W}_a \vect{h}_s\\
\vect{v}_a ^{\top}tahn(\vect{W}_a[\vect{\tilde{h}}_t;\vect{h}_s])\\
\end{cases}
\end{equation}

At the end, a global context vector $\vect{c}^{(t)}$ is computed as the weighted average, according to $\vect{a}_t$ over all source states:

\begin{equation}
\vect{c}^{(t)} = \sum_{s} \vect{a}_{ts}\vect{h}^{(s)}
\end{equation}




\chapter{Encontrando os Padrões Irregulares}
\label{ch:03-dialog-systems}

 In this chapter, we will review some techniques to produce dialogs together with some evaluation strategies. For brevity's sake, we decided not to give a full historic presentation of the field of dialog generation. Hence, our review of the techniques will be deep learning oriented.

\section{Uso do Encoder Decoder para tarefas linguisticas (?)}
\label{ch:03-gen}

\section{Resultados e Discussão}
\label{ch:03-results}
 
\chapter{Future Steps}
\label{ch04:FutureSteps}

The previous chapters resume what we have done so far: we grasp the theoretical framework to formulate this NLP problem; we reviewed the literature on dialog generation; we built a software workbench to perform different experiments; and we have isolated a specific problem not sufficiently addressed in the literature: logical reasoning for dialog agents.

In July we will present this work at one summer school organized by the company DeepMind in Europe \url{https://tmlss.ro/} and so we expect to gather more feedback on our research.

To address our research proposal we decided to formulate the following next steps:

\begin{itemize}
\item Apply regularization strategies on the available models to overcome the reported overfitting problem. 
\item Finish the Entailment-QA corpus to have a fine grain analysis of the result that we are seeing on the SICK corpus.
\item Explore the different extensions for all mentioned models.
\item Explore new models not mentioned here, like Dynamic Memory Networks \cite{KumarISBEPOGS15} and the models using the Memory Attention and Composition (MAC) cell \cite{Manning18}.
\item Create a visual version of the Entailment-QA to test logical inference with images.
\item There is a different literature that frames the dialog problem as an MDP (Markovian Decision Process) and a POMDP (Partially Observable Markovian Decision Process) applying different techniques of reinforcement learning (a recent example is \cite{Li:2016}). It is fruitful to investigate if these techniques can help our research.
\item One of the main focused here is model comparison. It would be fruitful if we could use the available literature  on the theory of comparing models (e.g.,  \cite{BenavoliCDZ17}) to refine our analysis.
\end{itemize}


\section{Work Plan}
\label{sec:work-plan}

Here, we use a visual tool to display the scheduling of future and past activities. This serve as a sanity check to verify if the
proposed goals can be realistically achieved.

\begin{table}[ht!]
  \center
  \begin{tabular}{|c|c|c|c|c|c|c|c|c|c|}\hline
    & \multicolumn{2}{c|}{2016} & \multicolumn{2}{c|}{2017} & \multicolumn{2}{c|}{2018} & \multicolumn{2}{c|}{2019} & \multicolumn{1}{c|}{2020} \\ \cline{2-10}
    \raisebox{1.5ex}{Activity} & 1st & 2nd & 1st & 2nd & 1st & 2nd & 1st & 2nd & 1st \\ \hline \hline
    Courses & \cellcolor{green!45} & \cellcolor{green!45} & \cellcolor{green!45}  & \cellcolor{green!45}  &  &  & & & \\ \hline
    Teaching Assist. (PAE) & & &  &  &\cellcolor{green!45}  &  & & & \\ \hline
    Bibliographic Review & \cellcolor{green!45} & \cellcolor{green!45} & \cellcolor{green!45} & \cellcolor{green!45} & \cellcolor{green!45} & &  &  & \\ \hline 
    Software Implementation & & & & \cellcolor{green!45} &\cellcolor{green!45}  & \cellcolor{blue!45} &\cellcolor{blue!45}&\cellcolor{blue!45}& \\ \hline
    Qualification Writing & & & & & \cellcolor{green!45} & &  & & \\ \hline
    Qualification Exam & & & & & \cellcolor{green!45} & & & & \\ \hline
    Finishing Entailment-QA task & & & & & & \cellcolor{blue!45} &  & & \\ \hline
    Visual Entailment-QA task & & & & & & \cellcolor{blue!45} & \cellcolor{blue!45}  & & \\ \hline
    Improve Training & & & & & & \cellcolor{blue!45} &  & & \\ \hline
    Adding new models & & & & & & \cellcolor{blue!45} &\cellcolor{blue!45}  & & \\ \hline
    Reinforcement Learning Methods & & & & & & &\cellcolor{blue!45}  &\cellcolor{blue!45} & \\ \hline
    Model Comparison Theory & & & & & & &\cellcolor{blue!45}  &\cellcolor{blue!45} & \\ \hline
    Thesis Writing & & & & & & & & \cellcolor{blue!45} & \\ \hline
    Thesis Defense & & & & & & & & & \cellcolor{blue!45}  \\ \hline
  \end{tabular}
  \caption{Schedule for the PhD program. Completed activities are shown in green, while future activities are in blue.}
  \label{tab:schedule}
\end{table}

% % cabeçalho para os apêndices
% \renewcommand{\chaptermark}[1]{\markboth{\MakeUppercase{\appendixname\ \thechapter}} {\MakeUppercase{#1}} }
% \fancyhead[RE,LO]{}
% \appendix

% \include{proofs}

% \include{ape-previous-work}

% ---------------------------------------------------------------------------- %
% Bibliografia
\backmatter \singlespacing   % espaçamento simples
\bibliographystyle{references-style-plainnat-ime} % citação bibliográfica textual
\bibliography{references}  % associado ao arquivo: 'bibliografia.bib'

% ---------------------------------------------------------------------------- %
% Índice remissivo
% \index{TBP|see{periodicidade região codificante}}
% \index{DSP|see{processamento digital de sinais}}
% \index{STFT|see{transformada de Fourier de tempo reduzido}}
% \index{DFT|see{transformada discreta de Fourier}}
% \index{Fourier!transformada|see{transformada de Fourier}}

% \printindex   % imprime o índice remissivo no documento 

\end{document}
