\chapter{Próximos Passos}
\label{ch04:FutureSteps}


Os resultados obtidos até agora indicam que: \textit{1. Aumentar a base de treino parece aprimorar o treinamento de forma razoável}; \textit{2. Aumentar o grau de resolução para transcrição fonética nessa mesma base} 

In July we will present this work at one summer school organized by the company DeepMind in Europe \url{https://tmlss.ro/} and so we expect to gather more feedback on our research.

To address our research proposal we decided to formulate the following next steps:

\begin{itemize}
\item Apply regularization strategies on the available models to overcome the reported overfitting problem. 
\item Finish the Entailment-QA corpus to have a fine grain analysis of the result that we are seeing on the SICK corpus.
\item Explore the different extensions for all mentioned models.
\item Explore new models not mentioned here, like Dynamic Memory Networks \cite{KumarISBEPOGS15} and the models using the Memory Attention and Composition (MAC) cell \cite{Manning18}.
\item Create a visual version of the Entailment-QA to test logical inference with images.
\item There is a different literature that frames the dialog problem as an MDP (Markovian Decision Process) and a POMDP (Partially Observable Markovian Decision Process) applying different techniques of reinforcement learning (a recent example is \cite{Li:2016}). It is fruitful to investigate if these techniques can help our research.
\item One of the main focused here is model comparison. It would be fruitful if we could use the available literature  on the theory of comparing models (e.g.,  \cite{BenavoliCDZ17}) to refine our analysis.
\end{itemize}


\section{Cronograma}
\label{sec:work-plan}

Nesta seção, a tabela \ref{tab:schedule} serve de apoio visual para uma melhor compreensão sobre as atividades e objetivos realizados bem como atividades planejadas até o final do programa.

\begin{table}[ht!]
  \center
  \begin{tabular}{|c|c|c|c|c|}\hline
    {Atividade} & 2ºsem 2017 & 1ºsem 2018 & 2ºsem 2018 & 1ºsem 2019 \\ \hline \hline
    Disciplinas & \cellcolor{green!45} &\cellcolor{green!45} &  & \\ \hline
    Revisão Bibliográfica & \cellcolor{green!45} & \cellcolor{green!45} & \cellcolor{yellow!45} & \\ \hline 
    Modelagem Computacional & \cellcolor{green!45} & \cellcolor{green!45} & \cellcolor{yellow!45}& \\ \hline
    Publicações & & & \cellcolor{green!45} & \\ \hline
   Participação em Eventos da área &\cellcolor{green!45} & \cellcolor{green!45} & \cellcolor{yellow!45} & \cellcolor{orange!45}\\ \hline
    Escrita da Qualificação & &\cellcolor{green!45} &\cellcolor{green!45} &  \\ \hline
    Exame de Qualificação & & &\cellcolor{yellow!45} & \\ \hline
    Readequação de Corpus & & &\cellcolor{orange!45} &  \\ \hline
    Testes de Novos Modelos & & &\cellcolor{orange!45} &\\ \hline
    Testes Psico-linguísticos & & & &\cellcolor{orange!45} \\ \hline
    Análise Modelos Computacionais x Falantes & & & &\cellcolor{orange!45} \\ \hline
    Escrita Dissertação & & &\cellcolor{yellow!45} &\cellcolor{orange!45}\\ \hline
    Defesa & & & &\cellcolor{orange!45} \\ \hline
  \end{tabular}
  \caption{Cronograma. Atividades realizadas em verde, atividades sendo realizadas em amarelo e próximas atividades em laranja.}
  \label{tab:schedule}
\end{table}