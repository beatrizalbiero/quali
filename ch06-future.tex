\chapter{Próximos Passos}
\label{ch04:FutureSteps}


Como apontado na Seção \ref{sec:disc}, existem ainda algumas possibilidades para a produção de um modelo com melhores resultados para que a comparação com o teste psicolinguístico seja interessante.\\

A seguir destaco os próximos passos para viabilizar a execução do restante desta pesquisa:

\begin{itemize}
\item Aumentar a base para 5000 verbos (treino + teste), porém aumentar a resolução dos mesmos para nível fonético.
\item Criar uma base de teste controlada, ou seja, ter um registro do número de verbos irregulares de cada família na base de teste.
\item Testar modelos de Atenção. Esses modelos foram apresentados por Dzmitry Bahdanau, et al. no artigo “Neural Machine Translation by Jointly Learning to Align and Translate” %ref
. A Atenção apresenta-se como uma solução que pode melhorar o desempenho do Encoder-Decoder em sequências longas e é utilizada em tradução automática tanto para o alinhamento quanto para a tradução dos termos.
\item Explorar técnicas de regularização como Dropout e Recurrent Dropout.
\item Desenvolver e aplicar um teste psicolinguístico para registrar a intuição dos falantes quanto à regularização dos verbos.
\item Discutir os resultados obtidos pelo modelo computacional e compará-los com as observações do teste psicolinguístico.

\end{itemize}


\section{Cronograma}
\label{sec:work-plan}

Nesta seção, a Tabela \ref{tab:schedule} serve como referência para a análise das atividades e objetivos realizados bem como atividades planejadas até o final do programa.

\begin{table}[ht!]
  \center
  \begin{tabular}{|c|c|c|c|c|}\hline
    {Atividade} & 2ºsem 2017 & 1ºsem 2018 & 2ºsem 2018 & 1ºsem 2019 \\ \hline \hline
    Disciplinas & \cellcolor{green!45} &\cellcolor{green!45} &  & \\ \hline
    Revisão Bibliográfica & \cellcolor{green!45} & \cellcolor{green!45} & \cellcolor{yellow!45} & \\ \hline 
    Modelagem Computacional & \cellcolor{green!45} & \cellcolor{green!45} & \cellcolor{yellow!45}& \\ \hline
    Publicações & & & \cellcolor{green!45} & \\ \hline
   Participação em Eventos da área &\cellcolor{green!45} & \cellcolor{green!45} & \cellcolor{yellow!45} & \cellcolor{orange!45}\\ \hline
    Escrita da Qualificação & &\cellcolor{green!45} &\cellcolor{green!45} &  \\ \hline
    Exame de Qualificação & & &\cellcolor{yellow!45} & \\ \hline
    Readequação de Corpus & & &\cellcolor{orange!45} &  \\ \hline
    Testes de Novos Modelos & & &\cellcolor{orange!45} &\\ \hline
    Testes Psico-linguísticos & & & &\cellcolor{orange!45} \\ \hline
    Análise Modelos Computacionais x Falantes & & & &\cellcolor{orange!45} \\ \hline
    Escrita Dissertação & & &\cellcolor{yellow!45} &\cellcolor{orange!45}\\ \hline
    Defesa & & & &\cellcolor{orange!45} \\ \hline
  \end{tabular}
  \caption{Cronograma. Atividades realizadas em verde, atividades sendo realizadas em amarelo e próximas atividades em laranja.}
  \label{tab:schedule}
\end{table}